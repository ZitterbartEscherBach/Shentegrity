\documentclass{article}
\usepackage[utf8]{inputenc}
\usepackage{geometry}
\geometry{margin=1in}
\usepackage{array}
\usepackage{longtable}
\usepackage{booktabs}
\usepackage{pdflscape}

\begin{document}

\begin{landscape}

\section*{Lo Shu 2: Consciousness and the Tensegrity Field}

\renewcommand{\arraystretch}{1.4}

\begin{tabular}{|p{1.8cm}|p{3cm}|p{5.5cm}|p{8.5cm}|}
\hline
\textbf{Lo Shu Position} & 
\textbf{Core Concept} & 
\textbf{Definition} & 
\textbf{Meaning in the Tensegrity Framework} \\
\hline

1 & 
Brane & 
Dimensional Field & 
The foundational substrate of reality — membrane logic, extended field dynamics. \\
\hline

2 & 
Shen & 
Emergent Awareness & 
Awareness as a result of relational structure, not an object — a field phenomenon. \\
\hline

3 & 
Pathos & 
Emotional Charge \& Sensitivity & 
The affective resonance field — emotional tonality modulates perceptual and structural states. \\
\hline

4 & 
Jing & 
Biophysical Structure & 
Stored energy and form — material expression of constraint, mass, and tension. \\
\hline

5 & 
Logos & 
Patterned Ordering of Experience & 
Structural coherence — the internal grammar of sensation and cognition. \\
\hline

6 & 
Brain & 
Processing Interface & 
Neurocognitive organ of modulation — dynamic tension regulator between structure and perception. \\
\hline

7 & 
Ethos & 
Principle of Interaction & 
Embodied orientation toward coherence, value, and meaning — directs how perception becomes behavior. \\
\hline

8 & 
Bagua & 
Force Distribution Map & 
Framework for structured adaptability — maps tension and flow across nested systems. \\
\hline

9 & 
Chi & 
Field Movement \& Flow & 
The animated force in tensegrity — vibrational transmission across spatial constraints. \\
\hline

\end{tabular}

\end{landscape}

\end{document}
