\documentclass{article}
\usepackage[utf8]{inputenc}
\usepackage{geometry}
\geometry{margin=1in}
\usepackage{array}
\usepackage{longtable}
\usepackage{booktabs}
\usepackage{pdflscape}

\begin{document}

\begin{landscape}

\section*{Lo Shu 6: Triangles of Possibility — Structural \& Philosophical Tensegrity}

\renewcommand{\arraystretch}{1.4}

\begin{tabular}{|p{1.8cm}|p{4.5cm}|p{9.5cm}|}
\hline
\textbf{Lo Shu Position} & 
\textbf{Concept} & 
\textbf{Meaning in Structural \& Philosophical Tensegrity} \\
\hline

1 & 
Obtuse Isosceles & 
Stretched tension in two balanced anchors — a force-differentiated triangle balancing two fixed points with one extended force. \\
\hline

2 & 
Equilateral & 
Perfectly distributed tension — idealized tensegrity with no single dominating force; structural harmony. \\
\hline

3 & 
Right Scalene & 
Asymmetrical tension with orthogonal constraint — a triangle locked into a perpendicular structural force. \\
\hline

4 & 
Acute Scalene & 
Unstable asymmetry seeking form — a triangle where all forces pull toward equilibrium without resolution. \\
\hline

5 & 
Right Isosceles & 
Fixed symmetry with a perpendicular anchor — balanced yet constrained; classical architectural structuring. \\
\hline

6 & 
Foucault’s Bio-Power & 
The structuring of bodies and social power — tensegrity of biopolitics; how embodied form is regulated by force systems. \\
\hline

7 & 
Adorno’s Negative Dialectics & 
Structural contradictions as form-finding — the balancing force of contradiction within systems of tension. \\
\hline

8 & 
Obtuse Scalene & 
Stretched and unbalanced force alignment — the distorted echo of the equilateral; extreme differential. \\
\hline

9 & 
Golden Triangle & 
Proportional harmonic balance — ideal tensegrity form expressing structural resonance and recursive harmony. \\
\hline

\end{tabular}

\end{landscape}

\end{document}
