\documentclass{article}
\usepackage[utf8]{inputenc}
\usepackage{geometry}
\geometry{margin=1in}
\usepackage{array}
\usepackage{longtable}
\usepackage{booktabs}

\begin{document}

\section*{Extraordinary Vessels and Quantum Physics}

\renewcommand{\arraystretch}{1.5} % spacing between rows

\begin{tabular}{|p{4cm}|p{5cm}|p{6.5cm}|}
\hline
\textbf{Extraordinary Vessel} & \textbf{Quantum Role} & \textbf{Jing / Chi / Shen Connection} \\
\hline
Chong Mai (Thrusting Vessel) & 
Central Axis of Mass/Energy & 
Jing (Third Gen: Most Massive, Foundational Field Stability) \\
\hline
Ren Mai (Conception Vessel) & 
Matter Formation (How Quantum Fields Become Classical) & 
Jing → Chi transition (Structuring the Field into Flow) \\
\hline
Du Mai (Governing Vessel) & 
Spacetime Curvature \& Gravity & 
Jing as Mass → Gravity as Field Interaction \\
\hline
Dai Mai (Belt Vessel) & 
Spin Networks / Torsion Fields & 
Chi (Second Gen: Rotational, Intermediate Particles) \\
\hline
Yang Qiao Mai (Yang Motility Vessel) & 
Light, Electromagnetism, Movement & 
Chi (Carries Wave Properties \& Forces Between Jing \& Shen) \\
\hline
Yin Qiao Mai (Yin Motility Vessel) & 
Inertia, Decoherence, Fixation of Quantum States & 
Chi → Shen (Converting Fields into Classical States) \\
\hline
Yang Wei Mai (Yang Linking Vessel) & 
Entanglement \& Quantum Coherence & 
Shen (First Gen: Relational, Stabilized, Perceivable Particles) \\
\hline
Yin Wei Mai (Yin Linking Vessel) & 
Cognitive \& Informational Encoding & 
Shen as Observer-Effect Field (Fundamental Awareness of Space-Time) \\
\hline
\end{tabular}

\end{document}
