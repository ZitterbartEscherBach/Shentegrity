\documentclass{article}
\usepackage[utf8]{inputenc}
\usepackage{geometry}
\geometry{margin=1in}
\usepackage{array}
\usepackage{longtable}
\usepackage{booktabs}
\usepackage{pdflscape}
\usepackage{amsmath} % for \pi

\begin{document}

\begin{landscape}

\section*{Lo Shu 4: Biotensegrity Framework and Mathematical Correlates}

\renewcommand{\arraystretch}{1.4}

\begin{tabular}{|p{2cm}|p{4cm}|p{6cm}|p{6cm}|}
\hline
\textbf{Lo Shu Position} & 
\textbf{Concept} & 
\textbf{Structural Meaning in Biotensegrity} & 
\textbf{Mathematical / Tensegrity Connection} \\
\hline

1 & 
Pi Symbol ($\pi$, Zero with Vertical Line Through It) & 
The fundamental geometric constraint of living form. & 
Circularity, recursion, and the tensegrity balance of closed systems. \\
\hline

2 & 
Facade & 
Outer structure, boundary formation in tissues. & 
Topology—tensegrity defines surfaces and membranes. \\
\hline

3 & 
Genetics & 
The encoded tensegrity blueprint of life. & 
Information structure—how forces shape biological inheritance. \\
\hline

4 & 
Fascia & 
The living, dynamic tensegrity network of the body. & 
Continuous tension networks—force propagation through soft tissue. \\
\hline

5 & 
Memetics & 
Pattern replication at the cognitive/cultural level. & 
Fractal tensegrity—scalable, non-local transmission of information. \\
\hline

6 & 
Neijing Tu & 
The dynamic warp—structural tensegrity as a cognitive-energetic field. & 
Integration of physics and biology—self-organizing tensegrity. \\
\hline

7 & 
Epigenetics & 
Tissue-level adaptation, environmental response. & 
Adaptive tensegrity—real-time adjustments to structure. \\
\hline

8 & 
Time & 
The fourth dimension of tensegrity, regulating adaptation. & 
Spacetime as a flexible tensegrity field. \\
\hline

9 & 
Flatland & 
Projection, dimensional limitations in perception. & 
How living structures exist in multi-dimensional constraints. \\
\hline

\end{tabular}

\end{landscape}

\end{document}
