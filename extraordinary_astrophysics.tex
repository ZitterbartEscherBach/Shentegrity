\documentclass{article}
\usepackage[utf8]{inputenc}
\usepackage{geometry}
\geometry{margin=1in}
\usepackage{array}
\usepackage{longtable}
\usepackage{booktabs}

\begin{document}

\section*{Quantum Twins: Identical vs. Fraternal}

\renewcommand{\arraystretch}{1.5} % increase spacing between rows

\begin{tabular}{|p{4.2cm}|p{5.2cm}|p{5.2cm}|}
\hline
\textbf{Concept} & \textbf{Identical Twins (Pauli Exclusion, Jing Model)} & \textbf{Fraternal Twins (Yin-Yang, Chi Model)} \\
\hline
Initial State & 
Start from the same DNA (identical fermions). & 
Start from two separate fertilized eggs (non-identical, yet bound). \\
\hline
Quantum Analogy & 
Fermions (Pauli Exclusion)—must be different quantum states. & 
Bosons (Chi Flow)—can interact freely without exclusion. \\
\hline
Structural Logic & 
Differentiation through resistance. & 
Differentiation through relational balance. \\
\hline
Emergence & 
Structure exists because states must be distinct. & 
Structure exists because opposites arise together. \\
\hline
Yin-Yang Model & 
One twin becomes the "dominant" form (Yang) while the other adapts to contrast (Yin). & 
Each twin is fundamentally separate but linked—they shape each other dynamically. \\
\hline
\end{tabular}

\end{document}
