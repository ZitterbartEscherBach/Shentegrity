\documentclass{article}
\usepackage[utf8]{inputenc}
\usepackage{geometry}
\geometry{margin=1in}
\usepackage{array}
\usepackage{longtable}
\usepackage{booktabs}
\usepackage{pdflscape}

\begin{document}

\begin{landscape}

\section*{Lo Shu 8: Awareness Evolution and the Structure of Perception}

\renewcommand{\arraystretch}{1.4}

\begin{tabular}{|p{1.8cm}|p{3.2cm}|p{10.8cm}|}
\hline
\textbf{Lo Shu Position} & 
\textbf{Concept} & 
\textbf{Meaning in Awareness Evolution} \\
\hline

1 & 
Primordial Yin–Yang & 
The undifferentiated field — the pre-patterned potentiality of all perception, before duality becomes structured. \\
\hline

2 & 
Oxherd & 
Agent of interaction — the first movement toward structured perception; guiding force, practice, embodiment. \\
\hline

3 & 
Weaver & 
The architect of form — the structuring of relationships; the weave of pattern, emergence of complexity. \\
\hline

4 & 
Star Child & 
Pre-manifest awareness — the formless-yet-structured potential; pre-conscious recognition. \\
\hline

5 & 
Buddha & 
The structural realization of awareness — the alignment of Shen with structured perception. \\
\hline

6 & 
Lao Tzu & 
The flow of perception — Tao as the structural rhythm of awareness; self-organizing emergence. \\
\hline

7 & 
Observer & 
The self-referential eye — the awareness of awareness itself; structured recursive cognition. \\
\hline

8 & 
Einstein & 
Field cognition — the awareness of structure as a field-based interaction; relativity as an emergent perceptual framework. \\
\hline

9 & 
“Bach” & 
The harmonic closure — the recursive aesthetic, the self-organizing Strange Loop, the Shen–Tensegrity balance in cognition. \\
\hline

\end{tabular}

\end{landscape}

\end{document}
