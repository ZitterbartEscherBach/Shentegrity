\documentclass{article}
\usepackage[utf8]{inputenc}
\usepackage{geometry}
\geometry{margin=1in}
\usepackage{array}
\usepackage{longtable}
\usepackage{booktabs}
\usepackage{pdflscape}

\begin{document}

\begin{landscape}

\section*{Lo Shu 1: Structural Physics and M-Theory Foundations}

\renewcommand{\arraystretch}{1.4}

\begin{tabular}{|p{1.8cm}|p{4.5cm}|p{9.5cm}|}
\hline
\textbf{Lo Shu Position} & 
\textbf{Mathematical Concept} & 
\textbf{Meaning in Structural Physics \& M-Theory} \\
\hline

1 & 
Field Measure & 
Measurement of space as a continuous field — the foundation of all structured knowledge; space itself as a measurable entity. \\
\hline

2 & 
Millet & 
Discrete units \& counting systems — the structure of quantization; how space is divided into countable parts. \\
\hline

3 & 
Distribution & 
The allocation of quantity across space — how force, energy, or material spreads in a structured way; field theory. \\
\hline

4 & 
Short Width & 
Geometric proportioning \& scaling rules — how dimensions interact; self-similarity and proportional coherence across scales. \\
\hline

5 & 
Construction & 
How structures are built \& reinforced — the process of form-finding; tensegrity expressed through mathematical logic. \\
\hline

6 & 
Levies & 
Constraints \& force-balancing in structured systems — tension-compression networks in both physical and conceptual systems. \\
\hline

7 & 
Excess / Deficit & 
Balance \& disequilibrium in structured systems — the dialectic of structural tension; the reason systems move, shift, or self-correct. \\
\hline

8 & 
Rectangles & 
Grid-based structural stability — the first structured containment of space; Bagua logic, modularity, base frameworks. \\
\hline

9 & 
Right Triangles & 
Geometric constraints \& the relationship between dimensions — the fundamental unit of higher-dimensional structuring; Pythagorean theorem, trigonometry, non-Euclidean geometries. \\
\hline

\end{tabular}

\end{landscape}

\end{document}
