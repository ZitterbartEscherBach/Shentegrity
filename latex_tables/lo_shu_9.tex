\documentclass{article}
\usepackage[utf8]{inputenc}
\usepackage{geometry}
\geometry{margin=1in}
\usepackage{array}
\usepackage{longtable}
\usepackage{booktabs}
\usepackage{pdflscape}

\begin{document}

\begin{landscape}

\section*{Lo Shu 9: Structural Encoding and the Architecture of Form}

\renewcommand{\arraystretch}{1.4}

\begin{tabular}{|p{1.8cm}|p{3.8cm}|p{10.2cm}|}
\hline
\textbf{Lo Shu Position} & 
\textbf{Concept} & 
\textbf{Meaning in Structural Encoding} \\
\hline

1 & 
Base to the Zeroth Power & 
The pure potential of structure — all numbers to the zeroth power = 1. This represents the singularity point, before differentiation. \\
\hline

2 & 
Sagittal Plane & 
The vertical axis (depth, front–back, inner–outer) — the first spatial orientation of form; the mirror plane of symmetry. \\
\hline

3 & 
Base 60 & 
Time encoding (Sumerian, Babylonian, harmonic cycles) — the structuring of time as a harmonic and calendrical system. \\
\hline

4 & 
Transverse Plane & 
The rotational axis (top–bottom, energy flow) — the cutting plane of cyclical movement; wave-like transformations. \\
\hline

5 & 
Base 10 & 
The human counting system (decimal, metric, AI default) — the dominant numerical interface for structured cognition. \\
\hline

6 & 
Base 46 & 
Genetic encoding (human chromosomes) — the direct structural mapping of biological information. \\
\hline

7 & 
Base 64 & 
Information encoding (I Ching, genetic codons, computing) — fractal encoding across symbolic, organic, and digital systems. \\
\hline

8 & 
Base 8 & 
Octal encoding (computing, cyclical harmony, Bagua) — the structural octaves of perception and system design. \\
\hline

9 & 
Coronal Plane & 
The lateral axis (left–right, integration, split processing) — the final orientation; how form integrates into field-based cognition. \\
\hline

\end{tabular}

\end{landscape}

\end{document}
