\documentclass{article}
\usepackage[table]{xcolor}
\usepackage{array}
\usepackage{longtable}
\usepackage[margin=0.75in]{geometry}

\begin{document}

\section*{Unified Scalar Ontology: Shentegrity Model}

\small
\rowcolors{2}{gray!10}{white}

\begin{longtable}{|>{\raggedright\arraybackslash}p{4cm}|>{\raggedright\arraybackslash}p{3.5cm}|>{\raggedright\arraybackslash}p{8cm}|}
\hline
\rowcolor{black!20}
\textbf{Aspect} & \textbf{Definition / State} & \textbf{Description} \\
\hline
Definition & Scalar & A quantity with magnitude only, invariant under coordinate transformations. \\
Tensor Rank & 0 & Scalar is a 0th-rank tensor (base unit of geometric algebra). \\
Directionality & None & Scalars do not encode or respond to direction or orientation. \\
Relationality & Non-relational & Defined without reference to other entities; intrinsic to point/system. \\
Dimensionality & Carries Units & Physical scalars hold dimensions (e.g., J, K, kg), not just number. \\
Transformation & Invariant & Scalar value remains unchanged under rotation, translation, Lorentz boost. \\
Topological Role & Pointwise / Field & Can define a local property or vary over a domain (scalar field). \\
Ontological Use & Foundational & Basis for building vectors, tensors, Lagrangians. Ground of formal systems. \\
Micro vs Macro & Context-Dependent & Appears in both quantum and cosmic regimes; not scale-limited. \\
\hline
\end{longtable}

\end{document}
