\documentclass{article}
\usepackage[utf8]{inputenc}
\usepackage{geometry}
\geometry{margin=1in}
\usepackage{longtable}
\usepackage{pdflscape}
\usepackage{booktabs}

\begin{document}

\begin{landscape}

\section*{Neurostructure and Five Phases Mapping}

\renewcommand{\arraystretch}{1.3}

\begin{longtable}{|p{3.5cm}|p{4.5cm}|p{5.5cm}|p{3.5cm}|}
\hline
\textbf{Neuroscientific Structure} & 
\textbf{TCM \& Taoist Symbolism} & 
\textbf{Function \& Metaphor} & 
\textbf{Five Phase (Element)} \\
\hline
\endfirsthead

\hline
\textbf{Neuroscientific Structure} & 
\textbf{TCM \& Taoist Symbolism} & 
\textbf{Function \& Metaphor} & 
\textbf{Five Phase (Element)} \\
\hline
\endhead

Optic Chiasm & Crystal Palace, Nexus of Shen & Balancing vision, merging perception, clarity of mind & Water \\
\hline
Pineal Gland & Third Eye (Yin Tang, Ajna in Indian traditions) & Higher intuition, inner sight, Shen focus & Fire \\
\hline
Pituitary Gland & Master regulator of the endocrine system & "Celestial connection," homeostasis, spirit-body balance & Fire \\
\hline
Hypothalamus & Seat of regulation (Qi balancing center) & Controls autonomic function, governs Shen stability & Metal \\
\hline
Corpus Callosum & Bridge between Yin \& Yang (Left-Right hemispheres) & Unites logic (Yang) and intuition (Yin) & Earth \\
\hline
\end{longtable}

\end{landscape}

\end{document}
