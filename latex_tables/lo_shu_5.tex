\documentclass{article}
\usepackage[utf8]{inputenc}
\usepackage{geometry}
\geometry{margin=1in}
\usepackage{array}
\usepackage{longtable}
\usepackage{booktabs}
\usepackage{pdflscape}

\begin{document}

\begin{landscape}

\section*{Lo Shu 5: Biotensegrity Types and Structural Balance}

\renewcommand{\arraystretch}{1.4}

\begin{tabular}{|p{1.8cm}|p{4.5cm}|p{9.5cm}|}
\hline
\textbf{Lo Shu Position} & 
\textbf{Biotensegrity Type} & 
\textbf{Meaning in Structural Balance} \\
\hline

1 & 
Biotensegrity & 
The body as a force distribution system — the foundational tensegrity of living tissue: fascia, bones, cells, and neural networks. \\
\hline

2 & 
Dynamic & 
Tensegrity in motion — how forces redistribute in real-time to maintain systemic balance. \\
\hline

3 & 
Hierarchical & 
Layered, scale-based stability — the structure of nested tensegrities in fractals, ecosystems, and cognitive networks. \\
\hline

4 & 
Form-Finding & 
The evolution of structure — how tensegrity naturally optimizes for efficiency, adaptability, and emergent form. \\
\hline

5 & 
Kinematic & 
Movement-based force balancing — governs the interaction between motion and structure across scales. \\
\hline

6 & 
Tessellated & 
Self-repeating, patterned force networks — grid-like force balance (e.g., Voronoi, hexagonal, and crystalline patterns). \\
\hline

7 & 
Positional & 
Fixed-point constraint balancing — how anchoring and spatial limitations affect system-wide force distribution. \\
\hline

8 & 
Knowledge-Based & 
Information as a structural field — the architecture of cognition, memory, and meaning as structural tension networks. \\
\hline

9 & 
Modular & 
Tensegrity as an adaptive system — modular units self-assembling and reorganizing across dynamic environments. \\
\hline

\end{tabular}

\end{landscape}

\end{document}
