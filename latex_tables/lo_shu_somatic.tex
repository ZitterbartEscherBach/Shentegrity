\documentclass{article}
\usepackage[utf8]{inputenc}
\usepackage{geometry}
\geometry{margin=1in}
\usepackage{array}
\usepackage{longtable}
\usepackage{booktabs}
\usepackage{pdflscape}
\usepackage{amsmath}

\begin{document}

\begin{landscape}

\section*{Somatic Lo Shu Mapping: Mathematics, Anatomy, and Tensegrity}

\renewcommand{\arraystretch}{1.4}

\begin{tabular}{|p{1.5cm}|p{2.8cm}|p{3.8cm}|p{3.8cm}|p{5.2cm}|p{5.2cm}|}
\hline
\textbf{Lo Shu} & 
\textbf{Human Opening} & 
\textbf{Nine Chapters (Mathematics)} & 
\textbf{Biotensegrity Concept} & 
\textbf{Tensegrity \& Structural Meaning} & 
\textbf{Mathematical Connection} \\
\hline

1 & 
Urogenital & 
Field Measurement (Geometry, area) & 
$\pi$ Symbol (⊘, Fundamental Form Constraint) & 
The root tensegrity—geometric first principle of structure. & 
Circularity, recursion, and the tensegrity balance of closed systems. \\
\hline

2 & 
Anus & 
Millet \& Rice (Proportions, fractions) & 
Facade (Outer Structural Layer) & 
Tensegrity defines outer membranes, boundaries, and form constraints. & 
Topology—how surfaces maintain dynamic balance. \\
\hline

3 & 
Mouth & 
Distribution by Proportion (Ratios, allocation) & 
Genetics & 
The encoded tensegrity blueprint of life—inheritance patterns of form. & 
Information theory—how forces shape biological structure. \\
\hline

4 & 
Left Nostril & 
Short Width (Scaling, division) & 
Fascia (Connective Network) & 
Living tensegrity matrix that propagates force and adapts to stress. & 
Continuous tension networks—force propagation through soft tissue. \\
\hline

5 & 
Right Nostril & 
Construction Engineering (Volumes, 3D structure) & 
Memetics & 
Scaling principles in cognition—how ideas and form interact in structural space. & 
Fractal tensegrity—non-local transmission of structural information. \\
\hline

6 & 
Left Ear & 
Levies (Optimization, distribution systems) & 
Neijing Tu (Dynamic Warp Field) & 
Biological and cognitive tensegrity—how the body and consciousness structure themselves. & 
Self-organizing tensegrity—feedback loops in force-balancing networks. \\
\hline

7 & 
Right Ear & 
Excess \& Deficit (Iterative solving, negative values) & 
Epigenetics (Tissue Adaptation) & 
Adaptive tensegrity—real-time physiological and structural adjustments. & 
Nonlinear constraints—balancing form under external forces. \\
\hline

8 & 
Left Eye & 
Rectangles (Linear Algebra, Matrix Theory) & 
Time & 
Tensegrity in four dimensions—structural adaptation over time. & 
Spacetime as a flexible tensegrity grid. \\
\hline

9 & 
Right Eye & 
Right Triangles (Pythagorean Theorem) & 
Flatland (Dimensional Constraint) & 
Projection, perception, and the limits of tensegrity-based awareness. & 
How living structures exist in multi-dimensional constraints. \\
\hline

\end{tabular}

\end{landscape}

\end{document}
