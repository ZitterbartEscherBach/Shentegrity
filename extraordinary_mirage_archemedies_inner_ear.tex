\documentclass{article}
\usepackage[margin=1in, landscape]{geometry}
\usepackage{array}
\usepackage{longtable}
\usepackage{lscape}
\usepackage{graphicx}
\usepackage{caption}
\captionsetup[table]{skip=10pt}

\begin{document}
\begin{landscape}

\begin{longtable}{>{\raggedright\arraybackslash}p{3.5cm} >{\raggedright\arraybackslash}p{4cm} >{\raggedright\arraybackslash}p{5.2cm} >{\raggedright\arraybackslash}p{4cm}}
\caption{Extraordinary Archimedean Middle Ear Mirage} \\
\textbf{System} & \textbf{Archimedean Principle} & \textbf{Scientific Function} & \textbf{Anatomical Structure} \\
\hline
\endfirsthead

\textbf{System} & \textbf{Archimedean Principle} & \textbf{Scientific Function} & \textbf{Anatomical Structure} \\
\hline
\endhead

Ossicles (Lever System) & Law of the Lever & Amplifies mechanical force to overcome impedance mismatch (air-to-fluid) & Malleus → Incus → Stapes → Oval Window \\

Eustachian Tube & Hydrostatic Pressure Equilibrium & Equalizes middle ear pressure with external atmosphere to prevent tympanic distortion & Connects middle ear to nasopharynx \\

Mastoid Air Cells & Principle of Buoyancy & Maintain pneumatic equilibrium, buffer pressure changes, ventilate tympanic cavity & Air-filled cavities posterior to ear \\

Tympanic Membrane & Wave Mechanics / Mechanical Coupling & Transduces airborne sound waves into mechanical vibrations & Eardrum (membrane at canal terminus) \\

Cochlear Fluid Dynamics & Hydraulic Displacement \& Energy Transfer & Converts mechanical displacement into traveling waves, initiating neural transduction & Stapes at Oval Window, pressure relieved at Round Window \\

\end{longtable}

\end{landscape}
\end{document}
