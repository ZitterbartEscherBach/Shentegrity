\documentclass{article}
\usepackage[utf8]{inputenc}
\usepackage{geometry}
\geometry{margin=1in}
\usepackage{array}
\usepackage{longtable}
\usepackage{booktabs}
\usepackage{pdflscape}

\begin{document}

\begin{landscape}

\section*{Lo Shu 3: Openings of the Body — Function and Structural Meaning}

\renewcommand{\arraystretch}{1.4}

\begin{tabular}{|p{1.8cm}|p{4.5cm}|p{9.5cm}|}
\hline
\textbf{Lo Shu Position} & 
\textbf{Opening} & 
\textbf{Function \& Structural Meaning} \\
\hline

1 & 
Urogenital & 
Root gateway — reproductive and excretory opening; the base-level exchange point: creation, elimination, survival, primal energy. \\
\hline

2 & 
Anus & 
Digestive exit — the completion of processing; final release of what remains after full transformation. \\
\hline

3 & 
Mouth & 
Speech, ingestion, breath control — the input/output portal of both nutrition and meaning: eating, speaking, singing. \\
\hline

4 & 
Left Nostril & 
Lunar (Yin) breathing pathway — cool, calming, inward-focused breath channel. \\
\hline

5 & 
Right Nostril & 
Solar (Yang) breathing pathway — warm, energizing, outward-focused breath channel. \\
\hline

6 & 
Left Ear & 
Reception of sound and vibration (Yin hearing) — passive intake of auditory frequency; associated with deep listening. \\
\hline

7 & 
Right Ear & 
Active hearing and cognitive sound processing (Yang hearing) — structural pathway for meaning extraction through sound. \\
\hline

8 & 
Left Eye & 
Perception through reflective, intuitive vision — the Yin aspect of sight; form, pattern, and relational awareness. \\
\hline

9 & 
Right Eye & 
Perception through focused, direct vision — the Yang aspect of sight; analytical clarity and direct interpretation. \\
\hline

\end{tabular}

\end{landscape}

\end{document}
