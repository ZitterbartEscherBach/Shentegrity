\documentclass{article}
\usepackage[utf8]{inputenc}
\usepackage{geometry}
\geometry{margin=1in}
\usepackage{array}
\usepackage{longtable}
\usepackage{booktabs}
\usepackage{pdflscape}

\begin{document}

\begin{landscape}

\section*{Lo Shu 7: Perceptual Spectrum and Physical Structuring}

\renewcommand{\arraystretch}{1.4}

\begin{tabular}{|p{1.8cm}|p{4.8cm}|p{9.2cm}|}
\hline
\textbf{Lo Shu Position} & 
\textbf{Concept} & 
\textbf{Meaning in Perceptual \& Physical Structuring} \\
\hline

1 & 
Relativity & 
The large-scale structure of spacetime — the warping of space and time as an emergent field. \\
\hline

2 & 
Ultraviolet Light & 
Beyond visible, high-energy field — the high-frequency, high-energy edge of optical perception. \\
\hline

3 & 
Audible Range & 
The primary range of human sound perception — the structured waveform space where human hearing is tuned. \\
\hline

4 & 
Ultra-Audible (Above Human Hearing) & 
Beyond perceptual hearing, high-frequency vibration — higher-ordered wave interactions that still affect structure (e.g., ultrasound, harmonic overtones). \\
\hline

5 & 
Newtonian Physics & 
The classical mechanics anchor — the fixed, force-based, material structure layer. \\
\hline

6 & 
Infrared & 
Heat-based, below visual perception — the lower-energy edge of optical perception; thermal structuring. \\
\hline

7 & 
ROYGBIV (Visible Spectrum) & 
The structured field of color perception — the full perceptual breakdown of optical waves into discrete bands. \\
\hline

8 & 
Infra-Audible (Below Human Hearing) & 
Long-wave, structural sound vibrations — the low-frequency, deep resonance field; subsonic structuring. \\
\hline

9 & 
Quantum Mechanics & 
The small-scale structure of reality — the wavefunction, probability, and uncertainty layer; the inverse of relativity (1). \\
\hline

\end{tabular}

\end{landscape}

\end{document}
