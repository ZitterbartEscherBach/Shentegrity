\documentclass{article}
\usepackage[utf8]{inputenc}
\usepackage{geometry}
\geometry{margin=1in}
\usepackage{array}
\usepackage{longtable}
\usepackage{booktabs}
\usepackage{pdflscape}

\begin{document}

\begin{landscape}

\section*{Chinese Mathematics and Tensegrity (Numerical REO Alignment)}

\renewcommand{\arraystretch}{1.4}

\begin{tabular}{|p{3.2cm}|p{3.5cm}|p{3.5cm}|p{6.5cm}|}
\hline
\textbf{Math} & 
\textbf{Method / Definition} & 
\textbf{Tensegrity} & 
\textbf{Meaning in Structure} \\
\hline

Field Measurement & 
Geometry & 
BioTensegrity & 
Equilibrium balancing; biological tensegrity logic. \\
\hline

Millet \& Rice & 
Proportions & 
Dynamic & 
How tension distributes dynamically. \\
\hline

Distribution by Proportion & 
Ratios & 
Hierarchical & 
Nested tensegrity forces; fractal relationships. \\
\hline

Short Width & 
Roots and Fractions & 
Form-Finding & 
Compression-expansion balance in tensegrity. \\
\hline

Construction Engineering & 
Solids & 
Kinematic & 
Movement and redirection of structural loads. \\
\hline

Fair Taxes & 
Linear Equations & 
Tessellated & 
Grid-like structure; spatial tiling of forces. \\
\hline

Excess \& Deficit & 
Unknowns & 
Position-Based & 
Dynamic stability via shifting weights. \\
\hline

Rectangular Arrays & 
Matrices & 
Knowledge-Based & 
Least-constraint solutions; optimization and systemic clarity. \\
\hline

Right Triangles & 
Pythagorean Theorem & 
Modular & 
Foundational triangle logic; structural coherence through modularity. \\
\hline

\end{tabular}

\end{landscape}

\end{document}
