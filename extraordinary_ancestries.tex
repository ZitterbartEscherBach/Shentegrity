\documentclass{article}
\usepackage[utf8]{inputenc}
\usepackage{geometry}
\geometry{margin=1in}
\usepackage{array}
\usepackage{longtable}
\usepackage{booktabs}

\begin{document}

\section*{Standard Model and the Jing–Chi–Shen Framework}

\renewcommand{\arraystretch}{1.5} % increases row height

\begin{tabular}{|p{3.8cm}|p{3.8cm}|p{3.8cm}|p{4.2cm}|}
\hline
\textbf{Standard Model} & \textbf{Jing (Essence, Structure)} & \textbf{Chi (Flow, Interaction)} & \textbf{Shen (Emergent Awareness, Refinement)} \\
\hline
First Generation (Lightest, Most Stable, Classical) & 
Up/Down Quarks, Electron, Electron Neutrino → These are the building blocks of matter, forming stable atoms. & 
Classical kinetic interactions (electromagnetism, chemistry). & 
Perceivable, embodied existence. \\
\hline
Second Generation (More Mass, Less Stable, Transient) & 
Charm/Strange Quarks, Muon, Muon Neutrino → Heavier, decays faster, but still has structure. & 
Higher energy interactions, intermediary between stable matter and ephemeral states. & 
Wave-like, transitional, links structure and emergence. \\
\hline
Third Generation (Most Massive, Least Stable, Quantum Field-Like) & 
Top/Bottom Quarks, Tau, Tau Neutrino → The most massive, shortest-lived particles. & 
Closest to quantum field interactions, raw force carriers. & 
Quantum-level emergence, existing for only brief moments but influencing the entire structure. \\
\hline
\end{tabular}

\end{document}
