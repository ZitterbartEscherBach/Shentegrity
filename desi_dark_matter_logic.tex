\documentclass{article}
\usepackage[margin=1in]{geometry}
\usepackage{booktabs}
\usepackage{tabularx}
\usepackage{ragged2e}
\renewcommand{\arraystretch}{1.5}

\begin{document}

\section*{Extraordinary Archimedean Middle Ear Mirage}

\begin{tabularx}{\textwidth}{>{\raggedright\arraybackslash}p{3.3cm} >{\raggedright\arraybackslash}p{4.3cm} >{\raggedright\arraybackslash}X}
\toprule
\textbf{System} & \textbf{Archimedean Principle} & \textbf{Scientific Function \& Anatomical Structure} \\
\midrule

Ossicular Lever System & Law of the Lever & Amplifies mechanical force to overcome impedance mismatch (air-to-fluid); transmits sound from tympanic membrane to malleus, incus, stapes \textrightarrow{} \textbf{oval window}. \\

Eustachian Tube & Hydrostatic Pressure Equilibrium & Equalizes middle ear pressure with atmospheric pressure; prevents tympanic distortion and connects to nasopharynx. \\

Mastoid Air Cell System & Principle of Buoyancy & Maintains pneumatic equilibrium, buffers pressure change, and ventilates tympanic cavity. \\

Tympanic Membrane & Wave Transmission via Mechanical Interface & Converts airborne sound waves into mechanical vibration of ossicles; initiates conduction path. \\

Oval \& Round Windows & Hydraulic Communication & Transduces mechanical energy into pressure waves in cochlear fluids; round window absorbs residual energy. \\

Cochlear Fluid Waves & Fluid Displacement Principle & Movement of stapes displaces perilymph in \textit{scala vestibuli}; initiates basilar membrane wave (frequency-dependent). \\

Hair Cells & Mechanical-to-Neural Transduction & Stereocilia deflection triggers ion channels; converts motion into electrical signals. \\

Basilar Membrane & Spatial Resonance (Tonotopy) & Frequency-to-place mapping from base (high frequencies) to apex (low frequencies); defines cochlear tuning. \\

Inner Ear Architecture & Contained Fluid Mechanics & Encapsulated labyrinth enables pressure wave propagation with minimal loss; maintains ionic separation between endolymph and perilymph. \\
\bottomrule
\end{tabularx}

\end{document}
