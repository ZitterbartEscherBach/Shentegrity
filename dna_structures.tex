\documentclass{article}
\usepackage[utf8]{inputenc}
\usepackage{geometry}
\geometry{margin=1in}
\usepackage{longtable}
\usepackage{pdflscape}
\usepackage{booktabs}

\begin{document}

\begin{landscape}

\section*{DNA Structural Forms by Complexity Rank}

\renewcommand{\arraystretch}{1.3}

\begin{longtable}{|p{2.2cm}|p{2.5cm}|p{2.2cm}|p{2.2cm}|p{4.5cm}|p{5.5cm}|}
\hline
\textbf{Rank} & 
\textbf{DNA Form} & 
\textbf{Handedness} & 
\textbf{Base Pairs per Turn} & 
\textbf{Key Features} & 
\textbf{Biological Relevance} \\
\hline
\endfirsthead

\hline
\textbf{Rank} & 
\textbf{DNA Form} & 
\textbf{Handedness} & 
\textbf{Base Pairs per Turn} & 
\textbf{Key Features} & 
\textbf{Biological Relevance} \\
\hline
\endhead

1 (Least Complex) & B-DNA & Right & ~10.5 & Canonical Watson-Crick DNA, most stable & Most common DNA form in cells \\
\hline
2 & A-DNA & Right & ~11 & More compact, deep major groove & Found in low-hydration conditions and RNA-DNA hybrids \\
\hline
3 & C-DNA & Right & ~9.3–9.6 & Similar to B-DNA but with minor structural changes & Observed in low-humidity environments \\
\hline
4 & D-DNA & Right & ~8 & Wider, lacks guanine, irregular helix & Seen in some synthetic sequences \\
\hline
5 & E-DNA & Right & Variable & Extended form of B-DNA, slightly overstretched & Observed in experimental conditions \\
\hline
6 & Z-DNA & Left & ~12 & Zigzag backbone, GC-rich regions, left-handed helix & Involved in gene regulation and immune response \\
\hline
7 & S-DNA & Right & Variable & Super-stretched DNA under tension & Studied in biophysics, DNA elasticity research \\
\hline
8 & G-DNA (G-Quadruplex DNA) & Variable & Variable & Stacked G-tetrads, stabilized by Hoogsteen pairing & Found in telomeres, linked to cancer and aging \\
\hline
9 (Most Complex) & P-DNA & Parallel-stranded & Variable & Highly stretched, bases flipped outward & Rare, seen in high-stretch conditions \\
\hline

\end{longtable}

\end{landscape}

\end{document}
